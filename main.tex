\documentclass{beamer}

% Use Unipd as theme, with options:
% - pageofpages: define the separation symbol of the footer page of pages (e.g.: of, di, /, default: of)
% - logo: position another logo near the Unipd logo in the title page (e.g. department logo), passing the second logo path as option 
\usetheme[pageofpages=of,logo=dei_logo.png]{Unipd}

\title{Presentations using the Unipd theme}
\subtitle{Demonstrating how to use the Unipd theme}
\author[Pippo, Pluto]{G.~Pippo \and P.~Pluto}

\date{April 20, 2000}

% The next block of commands puts the table of contents at the beginning of each section and highlights the current section
\AtBeginSection[]
{
  \begin{frame}
    \frametitle{Table of Contents}
    \tableofcontents[currentsection]
  \end{frame}
}



\begin{document}

% Make the title page
\frame{\titlepage}

% Insert the general toc
\begin{frame}{Table of Contents}
    \tableofcontents    
\end{frame}

\section{First section}

    \begin{frame}{First section}
        In this slide, some important text will be
        \alert{highlighted} because it's important.
        Here an ordered list:
        \begin{enumerate}
            \item First item
            \item Second item
        \end{enumerate}
        Here an unordered list: 
        \begin{itemize}
            \item One item
            \item Another item
        \end{itemize}
    \end{frame}
    
\section{Second section}

    \begin{frame}{Second section}
        \begin{block}{Block}
            Sample text in a normal block
        \end{block}
   
        \begin{alertblock}{Alert block}
            Sample text in an alert block
        \end{alertblock}    
        
         \begin{example}
            Sample text for an example
        \end{example}
    \end{frame}
    
\end{document}